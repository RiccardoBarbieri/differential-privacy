\chapter{Dimostrazioni}
\begin{proof}[\ref{eq:laplace_mechanism}]
\label{proof:laplace_mechanism}
Si definisca $x \in \mathbb{N}^k$ e $y \in \mathbb{N}^k$ tali che $||x - y||_1 \le 1$, e si consideri una funzione $f \colon \mathbb{N}^{|\mathcal{X}|} \to \mathbb{R}^k$. Si indichi con $p_x$ la funzione di densità di probabilità $\mathcal{M}_L(x,f,\varepsilon)$ e con $p_y$ la funzione di densità di probabilità $\mathcal{M}_L(y,f,\varepsilon)$. Comparando le due funzioni a un punto arbitrario $z \in \mathbb{R}^k$:
\begin{align*}
    \frac{p_x(z)}{p_y(z)} &= \prod_{i = 1}^{k}\left(\frac{\exp{\left(-\frac{\varepsilon |f(x)_i - z_i|}{\Delta f}\right)}}{\exp{\left(-\frac{\varepsilon |f(y)_i - z_i|}{\Delta f}\right)}}\right)\\
    &= \prod_{i = 1}^{k}\exp{\left(\frac{\varepsilon(|f(y)_i - z_i| - |f(x)_i - z_i|)}{\Delta f}\right)}\\
    \text{Applicando la disuguaglianza triangolare:}\\
    &\le \prod_{i = 1}^{k}\exp{\left(\frac{\varepsilon|f(x)_i-f(y)_i|}{\Delta f}\right)}\\
    &= \exp{\left(\frac{\varepsilon\cdot ||f(x) - f(y)||_1}{\Delta f}\right)}\\
    \text{Per la definizione di sensitività:}\\
    &\le e^\varepsilon
\end{align*}
\end{proof}

\begin{proof}[\ref{plot:normalized_inv_plrv}]
\label{proof:plrv_gaussian}
Si consideri un meccanismo $A$ e un parametro $\varepsilon \ge 0$, il minor parametro $\delta$ tale che il meccanismo $A$ è $(\varepsilon, \delta)$-DP è definito come:
\begin{equation*}
    \delta_S = \Pr[A(D_{in}) \in S] - e^\varepsilon \cdot \Pr[A(D_{out} \in S]
\end{equation*}
La definizione è soddisfatta se $\delta_S \le \delta$, quindi si definisce $\delta$:
\begin{equation*}
    \delta = \max_S(\Pr[A(D_{in}) \in S] - e^\varepsilon \cdot \Pr[A(D_{out} \in S])
\end{equation*}
Sono di interesse soltanto output $O$ tali che $\Pr[A(D_{in}) = O] > e^\varepsilon \cdot \Pr[A(D_{out}) = O]$, in quanto tutti gli altri output renderebbero $\delta_{max}$ più piccolo. L'insieme $S$ che minimizza questa quantità è:
\begin{equation*}
    S_{max} := \{O | \Pr[A(D_{in}) = O] > e^\varepsilon \cdot \Pr[A(D_{out}) = O]\}
\end{equation*}
Riscrivendo la definizione di $\delta$ precedente sostituendo con il nuovo insieme $S_{max}$ si ottiene:
\begin{align*}
    \delta &= \Pr[A(D_{in}) \in S_{max}] - e^\varepsilon \cdot \Pr[A(D_{out}) \in S_{max}]\\
    &= \sum_{O \in S_{max}} (\Pr[A(D_{in}) = O] - e^\varepsilon \cdot \Pr[A(D_{out}) = O])\\
    &= \sum_{O \in S_{max}} \Pr[A(D_{in}) = O]\left(1 - \frac{e^\varepsilon}{e^{\mathcal{L}_{D_{in},D_{out}}(O)}}\right)
\end{align*}
Invece di sommare soltanto gli output $O \in S_{max}$, possiamo riscrivere considerando tutti gli output $O \in S$ azzerando quelli che non appartengono a $S_{max}$.
\begin{equation*}
    \delta = \sum_O \Pr[A(D_{in}) = O] \max\left(0, 1 - \frac{e^\varepsilon}{e^{\mathcal{L}_{D_{in},D_{out}}(O)}}\right)
\end{equation*}
L'espressione sopra è un valore atteso pesato sulla probabilità che $A(D_{in}) = O$:
\begin{equation*}
\delta = \mathbb{E}_{O\sim A(D_{in})} \left[ \max\left(0, 1 - \frac{e^\varepsilon}{e^{\mathcal{L}_{D_{in},D_{out}}(O)}}\right)\right]
\end{equation*}
che corrisponde all'area tra la curva $e^{\mathcal{\varepsilon}}/e^{\mathcal{L}}$ e la retta $y = 1$.
\end{proof}